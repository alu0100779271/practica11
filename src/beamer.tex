\documentclass{beamer}
\usepackage[utf8]{inputenc}
\usepackage{graphicx}

\newtheorem{definicion}{Definición}
\newtheorem{ejemplo}{Ejemplo}

\begin{document}
%%%%%%%%%%%%%%%%%%%%%%%%%%%%%%%%%%%%%%%%%%%%%%%%%%%%%%%%%%%%%%%%%%%%%%%%%%%%%%%
\title[Presentación del número PI]{Número PI}
\author[Aidé Alicia Cordobés Betancor]{Aidé Alicia Cordobés Betancor}
\date[23-04-2014]{23 de abril de 2014}
%%%%%%%%%%%%%%%%%%%%%%%%%%%%%%%%%%%%%%%%%%%%%%%%%%%%%%%%%%%%%%%%%%%%%%%%%%%%%%%

%\usetheme{Madrid}
%\usetheme{Antibes}
%\usetheme{tree}
%\usetheme{classic}

%%%%%%%%%%%%%%%%%%%%%%%%%%%%%%%%%%%%%%%%%%%%%%%%%%%%%%%%%%%%%%%%%%%%%%%%%%%%%%%
  
%++++++++++++++++++++++++++++++++++++++++++++++++++++++++++++++++++++++++++++++
\begin{frame}
\titlepage
  \begin{small}
    \begin{center}
     Facultad de Matemáticas \\
     Universidad de La Laguna
    \end{center}
  \end{small}

\end{frame}
%++++++++++++++++++++++++++++++++++++++++++++++++++++++++++++++++++++++++++++++

%++++++++++++++++++++++++++++++++++++++++++++++++++++++++++++++++++++++++++++++
\begin{frame}
  \frametitle{Índice}
  \tableofcontents[pausesections]
\end{frame}
%++++++++++++++++++++++++++++++++++++++++++++++++++++++++++++++++++++++++++++++


\section{Un poco de historia}


%++++++++++++++++++++++++++++++++++++++++++++++++++++++++++++++++++++++++++++++
\begin{frame}

\frametitle{Un poco de historia}
La búsqueda del mayor número de decimales
del número $pi$ ha supuesto un esfuerzo constante
de numerosos científicos a lo largo de la historia.
Algunas aproximaciones históricas de pi son las
siguientes.

\end{frame}
%++++++++++++++++++++++++++++++++++++++++++++++++++++++++++++++++++++++++++++++

\section{Características matemáticas}

%++++++++++++++++++++++++++++++++++++++++++++++++++++++++++++++++++++++++++++++
\begin{frame}

\frametitle{Características matemáticas}
La búsqueda del mayor número de decimales
del número $pi$ ha supuesto un esfuerzo constante
de numerosos científicos a lo largo de la historia.
Algunas aproximaciones históricas de pi son las
siguientes.
\begin{block}{Características}
  \begin{itemize}
  \item
  $pi$ es la razón entre la longitud de cualquier circunferencia y la de su diámetro.
  \pause

  \item
  El área de un círculo unitario (de radio que tiene longitud 1, en el plano geométrico usual o plano euclídeo).
  \pause

  \item
  El menor número real x positivo tal que sin(x) = 0

  \end{itemize}
\end{block}

\end{frame}
%++++++++++++++++++++++++++++++++++++++++++++++++++++++++++++++++++++++++++++++

\section{Fórmulas}


%++++++++++++++++++++++++++++++++++++++++++++++++++++++++++++++++++++++++++++++
\begin{frame}
\frametitle{Fórmulas en las que aparece el número $pi$}
\begin{itemize}
  \item
  $S_n=a_1+\cdots + a_n = \sum_{i=1}^n a_i $
  \pause

  \item
  $\int_{x=0}^{\infty} x\,\text{e}^{-x^2}
\text{d}x=\frac{1}{2},\quad\text{e}^{i\pi}+1=0$
  \pause
  
  \item
  $\Vert x \Vert_2=1, \vert -7 \vert = 7,
m|n, m\mid n, <x,y>, \langle x, y\rangle$
  \pause
 
  \item
  $\frac{\text{d}}{\text{d}t}\left(\dfrac{\partial L}
{\partial\dot q_j}\right)-\frac{\partial L}
{\partial q_j}=0$
  \pause
   
  \item
  $\sqrt 2 = 1+\frac{1}{2+\frac{1}{2+
\frac{1}{2+\frac{1}{\ddots}}}} $
  \pause

  \end{itemize}
\end{frame}
%++++++++++++++++++++++++++++++++++++++++++++++++++++++++++++++++++++++++++++++


%++++++++++++++++++++++++++++++++++++++++++++++++++++++++++++++++++++++++++++++

\section{Bibliografía}
%++++++++++++++++++++++++++++++++++++++++++++++++++++++++++++++++++++++++++++++
\begin{frame}
  \frametitle{Bibliografía}

  \begin{thebibliography}{10}

    \beamertemplatebookbibitems
    \bibitem[Número PI, 2014]{PI}
    Número PI.
    (2014)
    {\small $http://es.wikipedia.org/wiki/N$}

    \beamertemplatebookbibitems
    \bibitem[Número PI, 2014]{PI}
    Número PI.
    (2014)
    {\small $http://centros5.pntic.mec.es/ies.de.bullas/dp/matema/conocer/numpi.htm$}

  \end{thebibliography}
\end{frame}

%++++++++++++++++++++++++++++++++++++++++++++++++++++++++++++++++++++++++++++++
\end{document}